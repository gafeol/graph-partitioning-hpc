\documentclass[unicode,11pt,a4paper,oneside,numbers=endperiod,openany]{scrartcl}

\usepackage{assignment}

\usepackage{amssymb}

\begin{document}

\setassignment
\setduedate{13 December 2019, 23:59}

\serieheader{High Performance Computing}{2019}{Student: FULL NAME}{Discussed with: FULL NAME}{Solution for Assignment 7}{}
\newline

\section*{Graph Partitioning}
In this assignment  you will implement various graph partitioning algorithms in Matlab and test these methods on a variety of 2D meshes.



\section{Install METIS 5.0.2, KaHIP  2.0, and the corresponding Matlab mex interface \punkte{10}}
\section{Implement various graph partitioning algorithms in Matlab \punkte{40}}
\begin{table}[h]
	\caption{Edge-cut Results}
	\centering
	\begin{tabular}{l|r|r|r|r|r} \hline\hline 
		Mesh             &  Coordinate           & Metis 5.0.2 &  KaHIP & Spectral & Inertial  \\ \hline
		grid5rec(8, 80)  &    8                   &  8         &   8       &   8      &   8        \\             
		grid5rec(80, 8)  &    8                   &  8         &           &          &           \\ 
		gridt(40)        &                        &            &           &          &           \\ 
		grid9(40)        &                        &            &           &          &           \\ 
		small            &                        &            &           &          &           \\
		Tapir            &                        &            &           &          &           \\ 
		Eppstein         &                        &            &           &          &           \\ 
		Airfoil          &                        &            &           &          &           \\ 
		cockroach(90)    &                        &            &           &          &           \\ \hline \hline
	\end{tabular}
	\label{table:edge}
\end{table}

\section{Visualize the graph partitioning \punkte{10}}
\section{Implement in Matlab the recursive $k$-way partitioning \punkte{10}}
\begin{table}[h]
	\caption{Edge-cut results for k-way partitioning and the airfoil mesh.}
	\centering
	\begin{tabular}{l|r|r|r|r|r} \hline\hline 
		Mesh            &  Coordinate           & Metis 5.0.2 & KaHIP & Spectral & Inertial  \\ \hline
		k=2             &                       &           &           &          &           \\             
		k=4             &                       &           &           &          &           \\ 
		k=8             &                       &           &           &          &           \\ 
		k=16            &                       &           &           &          &           \\ 
		k=32            &                       &           &           &          &           \\ \hline \hline
	\end{tabular}
	\label{table:kway}
\end{table}
\section{Partitioning of realistic large-scale FEM meshes \punkte{30}}
\begin{table}[h]
	\caption{Results for 2-way partitioning of the selected FEM mesh.}
	\centering
	\begin{tabular}{l|r|r} \hline\hline 
		Metric            &  Metis 5.0.2      & KaHIP \\ \hline
		Time (s)         &                       &                \\             
		Partition 1         &                       &          \\  
		Partition 2         &                       &         \\  
		Edge cut            &                       &       \\ \hline \hline
	\end{tabular}
	\label{table:FEMkway}
\end{table}

\end{document}
